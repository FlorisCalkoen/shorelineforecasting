%% This Beamer template is based on the one found here: https://github.com/sanhacheong/stanford-beamer-presentation, and edited to be used for Stanford ARM Lab

\documentclass[10pt]{beamer}
%\mode<presentation>{}

\usepackage{media9}
\usepackage{amssymb,amsmath,amsthm,enumerate}
\usepackage[utf8]{inputenc}
\usepackage{array}
\usepackage[parfill]{parskip}
\usepackage{graphicx}
\usepackage{caption}
\usepackage{subcaption}
\usepackage{bm}
\usepackage{amsfonts,amscd}
\usepackage[]{units}
\usepackage{listings}
\usepackage{multicol}
\usepackage{multirow}
\usepackage{tcolorbox}
\usepackage{physics}

% Enable colored hyperlinks
\hypersetup{colorlinks=true}

% The following three lines are for crossmarks & checkmarks
\usepackage{pifont}% http://ctan.org/pkg/pifont
\newcommand{\cmark}{\ding{51}}%
\newcommand{\xmark}{\ding{55}}%

% Numbered captions of tables, pictures, etc.
\setbeamertemplate{caption}[numbered]

%\usepackage[superscript,biblabel]{cite}
\usepackage{algorithm2e}
\renewcommand{\thealgocf}{}

% Bibliography settings
%\usepackage[style=ieee]{biblatex}
%\setbeamertemplate{bibliography item}{\insertbiblabel}
%\addbibresource{references.bib}

\usepackage[style=ieee, backend=biber]{biblatex}
\setbeamertemplate{bibliography item}{\insertbiblabel}
\addbibresource{references.bib}

%\usepackage{natbib}

% Glossary entries
\usepackage[acronym]{glossaries}
\newacronym{ML}{ML}{machine learning}
\newacronym{HRI}{HRI}{human-robot interactions}
\newacronym{RNN}{RNN}{Recurrent Neural Network}
\newacronym{LSTM}{LSTM}{Long Short-Term Memory}


\theoremstyle{remark}
\newtheorem*{remark}{Remark}
\theoremstyle{definition}

\newcommand{\empy}[1]{{\color{darkorange}\emph{#1}}}
\newcommand{\empr}[1]{{\color{cardinalred}\emph{#1}}}
\newcommand{\examplebox}[2]{
\begin{tcolorbox}[colframe=darkcardinal,colback=boxgray,title=#1]
#2
\end{tcolorbox}}

\usetheme{Stanford} 
\input{./style_files_stanford/my_beamer_defs.sty}
%\logo{\includegraphics[height=0.4in]{./style_files_stanford/SU_New_BlockStree_2color.png}}
\logo{\includegraphics[width=1.2in]{./images/deltares_pay-off_wit.eps}}

% commands to relax beamer and subfig conflicts
% see here: https://tex.stackexchange.com/questions/426088/texlive-pretest-2018-beamer-and-subfig-collide
\makeatletter
\let\@@magyar@captionfix\relax
\makeatother

\title[Future Shorelines]{Forecasting Shoreline Evolution Using Satellite-Derived Shoreline Positions}
%\subtitle{Subtitle Of Presentation}

%\beamertemplatenavigationsymbolsempty

\begin{document}

\author[Deltares]{
	\begin{tabular}{c} 
	\Large
	Floris Calkoen\\
    \footnotesize \href{mailto:floris@calkoen.nl}{floris@calkoen.nl}
\end{tabular}
\vspace{-4ex}}

\institute{
	\vskip 5pt
	\begin{figure}
		\centering
		\begin{subfigure}[t]{0.5\textwidth}
			\centering
%			\includegraphics[height=0.33in]{./style_files_stanford/stanford_logo_2.png}
			\includegraphics[height=0.33in]{./images/uvalogo_science.eps}
		\end{subfigure}%
		~ 
		\begin{subfigure}[t]{0.5\textwidth}
			\centering
			\includegraphics[height=0.33in]{./images/deltareslogo.eps}
		\end{subfigure}
	\end{figure}
	\vskip 5pt
	MSc-thesis Data Science\\
	University of Amsterdam\\
	\vskip 5pt
	Supervised by: \\ 
	Dr. Cristian Rodriguez Rivero\\ 
	Dr. Arjen Luijendijk \\ 
	MSc. Eti\"{e}nne Kras\\
	\vskip 3pt
}

% \date{June 15, 2020}
\date{\today}

\begin{noheadline}
\begin{frame}\maketitle\end{frame}
\end{noheadline}

\setbeamertemplate{itemize items}[default]
\setbeamertemplate{itemize subitem}[circle]

%\begin{frame}
%	\frametitle{Overview} % Table of contents slide, comment this block out to remove it
%	\tableofcontents % Throughout your presentation, if you choose to use \section{} and \subsection{} commands, these will automatically be printed on this slide as an overview of your presentation
%\end{frame}

\section{Introduction}
\begin{frame}[allowframebreaks]
\frametitle{Introduction}

\begin{figure}
        \centering
        \includegraphics[width=0.8\textwidth]{images/cap-ferret.png}
        \caption{L\`{e}ge Cap-Ferret: left is 1963; right is 2010.}
        \label{fig:cap-ferret-1}
\end{figure}

\framebreak

\begin{figure}
	\centering
	\includegraphics[width=0.8\textwidth]{images/cap-ferret2.png}
	\caption{Plage de la Point, Cap-Ferret}
	\label{fig:cap-ferret-2}
\end{figure}

\framebreak

\begin{figure}
	\centering
	\includegraphics[width=1\textwidth]{images/shorelinemonitor.pdf}
	\caption{Global hotspots of beach erosion and accretion, after \textcite{Luijendijk2018state}(2018)}
	\label{fig:erosion-trends}
\end{figure}

\framebreak

\begin{figure}
	\centering
	\includegraphics[width=1\textwidth]{images/shoreline-forecast.png}
	\caption{Projected long-term shoreline changes, after \textcite{vousdoukas2020sandy}(2020)}
	\label{fig:lr-forecast-changes}
\end{figure}

\framebreak

\begin{figure}
	\centering
	\includegraphics[width=1\textwidth]{images/sandengine.png}
	\caption{SDS positions, with linear trend, at the Sand-engine, Netherlands, after \textcite{Kras2019shoreline}}
	\label{fig:sand-engine-example}
\end{figure}

\end{frame}

\begin{frame}
\frametitle{Research objectives}
	\begin{itemize}
		\item <1->What forecasting techniques are suitable for predicting shoreline evolution from SDS position data? Considering:  
		\begin{itemize}
			\item<2-> Quality metrics
			\item<3-> Cost of computation
		\end{itemize}
		\item<4-> To what extent the can the performance of the different algorithms be explained by the physical dynamics that the time-series represents? 

	\end{itemize}
\end{frame}

\section{methodology}
\begin{frame}[allowframebreaks]
	\frametitle{Pre-processing}
	
  
%	


\begin{figure}
	\centering
	\includegraphics[width=0.8\textwidth]{images/ts-example.pdf}
	\caption{Raw time-series example, showing transect \texttt{BOX\_187\_090\_56}, south of the IJmuiden jetty, The Netherlands.}
	\label{fig:ts-example}
\end{figure}
%
\framebreak
	\begin{table}[ht]
	\centering
	\caption{Count, nan, and timespan statistics while gradually selecting the final dataset.}
	\label{tab:data-selection}
	\begin{tabular}{lllll}
		\hline
		\textbf{Selection}           & \textbf{Sites (n)} & \textbf{Nan's (n)} & \textbf{Nan's (\%)} & \textbf{T (yr)} \\
		\hline
		Raw data                     & 1780724              & 11841277           & 25.2                & 26.4             \\
		Sandy              & 666555               & 4305559            & 24.3                & 26.5             \\
		Drop sed. comp.    & 655167               & 4085411            & 23.3                & 26.8             \\
		Drop low shlines             & 638368               & 3577115            & 20.5                & 27.4             \\
		Drop err chngert.              & 637611               & 3560182            & 20.4                & 27.4             \\
		Drop err tspan.                 & 635648               & 3513763            & 20.1                & 27.5             \\
		Sample                       & 133684               & 1018698            & $\sim$20.1          & $\sim$27.5       \\
		Drop outliers                & 133684               & 1018698            & 30.0                & 25.4             \\
		Drop Nan's \textgreater 15\% & 50666                & 103754             & 6.6                 & 31.0            \\
		\hline
	\end{tabular}
\end{table}

\end{frame}

\begin{frame}
\frametitle{Forecasting algorithms}


%\examplebox{Traditional statistical approaches}{
%	\begin{minipage}[t]{0.48\linewidth}%
%	\begin{itemize}
%		\item <2-> Linear regression
%		\item <3-> Exponential smoothing
%		\item <4-> Double Exponential Smoothing
%		\item <5-> ARIMA
%	\end{itemize}
%    \end{minipage}
%	\hfill
%	\begin{minipage}[t]{0.48\linewidth}
%		\alert{Local models}
%	\end{minipage}
%}

\examplebox{Traditional statistical approaches}{
	\begin{columns}
	\column{0.48\linewidth}%
		\begin{itemize}
			\item <2-> Linear regression
			\item <3-> Exponential smoothing
			\item <4-> Double Exponential Smoothing
			\item <5-> ARIMA
		\end{itemize}

	\column<7->{0.48\linewidth}
		\alert{LOCAL MODELS}
	\end{columns}
}


\examplebox{Modern ML approaches}{
\begin{columns}
\column{0.48\linewidth}
\begin{itemize}
	\item <6-> DL LSTM seq2seq network
\end{itemize}
\column<7->{0.48\linewidth}
		\alert{GlOBAL MODEL}
\end{columns}

}
\end{frame}

\begin{frame}
\frametitle{Split \& evaluation}
\framesubtitle{Traditional methods}
	\begin{figure}
		\centering
		\includegraphics[height=0.5\textheight]{images/split.pdf}
		\caption{Schematic overview DL LSTM network}
		\label{fig:dlnetwork}
	\end{figure}

\examplebox{Quality metrics}{
\begin{columns}
	\column<2->{0.48\textwidth}
	\begin{itemize}
		\item MAE
		\item MSE
	\end{itemize}
	\column<2->{0.48\textwidth}
		\begin{itemize}
		\item RMSE
		\item MAPE
	\end{itemize}
\end{columns}
}
\end{frame}

\section{Results}

\begin{frame}
\frametitle{Linear regression}
	\begin{figure}
		\centering
		\includegraphics[width=0.8\textwidth]{images/lr-example.pdf}
		\caption{LR forecast for transect \texttt{BOX\_187\_090\_56}, south of the IJmuiden jetty, The Netherlands.}
		\label{fig:lr-example}
	\end{figure}
\examplebox{}{
\begin{equation}
\label{eq:lr}
y_t = \beta_o + \beta_1x_t + \epsilon_t
\end{equation}
}
\end{frame}

\begin{frame}
	\frametitle{Exponential smoothing}
	\begin{figure}
		\centering
		\includegraphics[width=0.8\textwidth]{images/es-example.pdf}
		\caption{ES forecast for transect \texttt{BOX\_158\_135\_31} with $\alpha = 0.81$, tidal-ebb delta, NW USA}
		\label{fig:es-example}
	\end{figure}

\examplebox{}{
\begin{equation}
\label{eq:es}
\hat{y}_t = a * y_t + (1 - \alpha) * \hat{y}_{t-1} 
\end{equation}
}
\end{frame}

\begin{frame}
	\frametitle{ARIMA}
	\begin{figure}
		\centering
		\includegraphics[width=0.8\textwidth]{images/arima-example.pdf}
		\caption{ARIMA ($p=1; q=0; d=0$) forecasting for transect \texttt{BOX\_149\_041\_61}, on a stretch of sandy coast in south-western China.}
		\label{fig:arima-example}
	\end{figure}

\examplebox{}{
	\begin{columns}
	\column{0.3\textwidth}
	p = AR order
	\column{0.3\textwidth}
	q = MA order
	\column{0.3\textwidth}
	d = differencing
	\end{columns}
}

\end{frame}

\begin{frame}
	\frametitle{DL LSTM network}
	\begin{figure}
		\centering
		\includegraphics[width=1\textwidth]{images/dlnetwork.pdf}
		\caption{Simplified schematic overview the LSTM seq2seq network}
		\label{fig:dlnetwork}
	\end{figure}
\end{frame}

\begin{frame}
	\frametitle{DL LSTM network}
	\begin{figure}
		\centering
		\includegraphics[width=0.8\textwidth]{images/lstm-example.pdf}
		\caption{LSTM forecast for transect \texttt{BOX\_106\_132\_44}, south of the IJmuiden jetty, The Netherlands.}
		\label{fig:lstm-example}
	\end{figure}
\examplebox{}{
\begin{equation}
\label{eq:mseloss}
l(x, y) = L \{l_1, \dots, l_n\}^T, l_n = (x_n - y_n)^2
\end{equation}  
}
\end{frame}

\begin{frame}[allowframebreaks]
\frametitle{Performance}
\examplebox{Quality metrics}{
\begin{table}[h]
	\centering
	\caption{Performance metrics of the forecasting algorithms.}
	\label{tab:metrics}
	\begin{tabular}{llllll}
		\hline
		& mse    & mae  & mape & rmse & \\ 
		\hline
		Linear regr.          & 4040.4 & 22.5 & 8.9  & 25.8 & \\
		Exp. Smoothing        & 2687.4 & 18.9 & 7.4  & 22.2 & \\
		Double exp. Smoothing & 6016.5 & 31.6 & 12.7 & 35.2 & \\
		ARIMA\footnote{ARIMA evaluated on subsample ($N=300$)}                 & 506.2  & 11.2 & 2.42 & -    & \\
		LSTM                  & 2641   & 24.1 & 5.9  & 27.4 & \\
		\hline
	\end{tabular}
\end{table}
}

\framebreak

%\examplebox{Computational cost }{
%\begin{columns}
%\column<2->{0.48\textwidth}
%\begin{description}
%	\item[LR]{14 s }
%	\item[ES]{6.75 s} 
%\end{description}
%
%\column<2->{0.48\textwidth}
%
%\begin{description}
%	\item[ARIMA]{2 hours}
%	\item[LSTM]{1.58 s} 
%\end{description}


\examplebox{Computational cost}{
	\begin{table}[h]
		\centering
		\caption{Computational cost for 10.000 forecasts.}
		\label{tab:cost}
		\begin{tabular}{llllll}
			\hline
			& LR    & ES  & \alert{ARIMA\footnote{\alert{ARIMA only on subsample($N=100$)}}} & LSTM & \\ 
			\hline
			Time          & 14.1s  & 7s & 2h  & 1.6s & \\
			\hline
		\end{tabular}
	\end{table}
}


\end{frame}

\section{Discussion}
\begin{frame}
\frametitle{Discussion}

	\begin{figure}
	\centering
	\begin{subfigure}[t]{0.5\textwidth}
		\centering
		\includegraphics[width=0.85\textwidth]{images/lr-example.pdf}
		\caption{LR}
	\end{subfigure}%
	~ 
	\begin{subfigure}[t]{0.5\textwidth}
		\centering
		\includegraphics[width=0.85\textwidth]{images/es-example.pdf}
		\caption{ES}
	\end{subfigure}
	\end{figure}

	\begin{figure}
	\centering
	\begin{subfigure}[t]{0.5\textwidth}
		\centering
		\includegraphics[width=0.85\textwidth]{images/arima-example.pdf}
		\caption{ARIMA}
	\end{subfigure}%
	~ 
	\begin{subfigure}[t]{0.5\textwidth}
		\centering
		\includegraphics[width=0.85\textwidth]{images/lstm-example.pdf}
		\caption{LSTM}
	\end{subfigure}
	\end{figure}	

\end{frame}

\section{Conlusion}
\begin{frame}
\frametitle{Conclusion}

\begin{itemize}[<+->]
	\item LR forecast most sites can be improved using more complex algorithms 
	\item Suitability forecasting algorithms depends on physical dynamics that site is subject to 
	\item ARIMA promising results at local modelling level 
	\item Overall (accuracy, computational cost \& room for improvement) LSTM most promising
\end{itemize}

\end{frame}



\section{Outlook}
\begin{frame}	
\frametitle{Outlook}

	\begin{itemize}[<+->]
		\item Implement locality into global LSTM seq2seq model $\rightarrow$ hybrid model
		\begin{itemize}
			\item proxy data
			\item time series statistics
		\end{itemize}   
		\item Time series classification prior to forecasting  
		\item Interactive application  
	\end{itemize}

\end{frame}


%\section{Introduction}
%% `[allowframebreaks]` can be used to have multiple slides in one frame, where the slides are continued with the suffix "(cont.)"; `[allowframebreaks]` can be used with `\framebreak` to manually break each frame into multiple slides
%\begin{frame}[allowframebreaks]
%\frametitle{Introduction}
%	Itemize example
%	\begin{itemize}
%		\item \cite{example2020citation}
%		\item Item 2
%        \begin{table}[]
%        \caption{Example of Table - Taxonomy of human intent prediction}
%        \label{tab:table_example}
%        \vspace{-.75cm}
%        \resizebox{0.95\textwidth}{!}{%
%        \begin{tabular}{|c|c|c|c|}
%        \hline
%        \multicolumn{2}{|c|}{\multirow{2}{*}{Human}} & \multicolumn{2}{c|}{\begin{tabular}[c]{@{}c@{}}Execution Strategy\\ (Action)\end{tabular}}                            \\ \cline{3-4} 
%        \multicolumn{2}{|c|}{}                       & \begin{tabular}[c]{@{}c@{}}Observer\\ Knows\end{tabular} & \begin{tabular}[c]{@{}c@{}}Observer\\ Unknown\end{tabular} \\ \hline
%        \multirow{2}{*}{\begin{tabular}[c]{@{}c@{}}Objective \\ Function\end{tabular}} &
%          \begin{tabular}[c]{@{}c@{}}Observer\\ Knows\end{tabular} &
%          \begin{tabular}[c]{@{}c@{}}All is Known (e.g. Ping Pong) \\ where both objective and actions are clear\end{tabular} &
%          \begin{tabular}[c]{@{}c@{}}Human Action Model is unclear\\  or suboptimal (e.g. chess)\end{tabular} \\ \cline{2-4} 
%         &
%          \begin{tabular}[c]{@{}c@{}}Observer\\ Unknown\end{tabular} &
%          \begin{tabular}[c]{@{}c@{}}Human action model is well known, \\ but objective is not (e.g. joy-riding in car \\ or free running, where destination\\  or direction is unclear)\end{tabular} &
%          \begin{tabular}[c]{@{}c@{}}Poor action model and objective\\  function (e.g. Poor / good cook, \\ no idea of final dish)\end{tabular} \\ \hline
%        \end{tabular}%
%        }
%        \end{table}
%
%		\item Tables can be referenced as Table  \ref{tab:table_example}
%	\end{itemize}
%	
%	\framebreak
%	
%	Example of a figure, shown in Figure \ref{fig:prob_formulation_scenario_1}.
%	
%	\begin{figure}
%        \centering
%        \includegraphics[width=0.8\textwidth]{images/prob_formulation_scenario_1.png}
%        \caption{Example Figure}
%        \label{fig:prob_formulation_scenario_1}
%    \end{figure}
%\end{frame}
%
%% This demonstrates a new section
%\section{Examples}
%% This demonstrates a single frame without framebreaks
%\begin{frame}{Example of Horizontal Subfigures}
%
%	\begin{figure}
%		\centering
%		\begin{subfigure}[t]{0.5\textwidth}
%			\centering
%			\includegraphics[width=0.9\textwidth]{images/stone2014fall_setup.png}
%			\caption{Single Kinect setup for fall prevention in elderly residence \cite{stone2014fall}}
%		\end{subfigure}%
%		~ 
%		\begin{subfigure}[t]{0.5\textwidth}
%			\centering
%			\includegraphics[width=\textwidth]{images/staranowicz2015easy_multiple_kinects.png}
%			\caption{Multiple Kinects calibration for fall prediction\cite{staranowicz2015easy}}
%		\end{subfigure}
%		\caption{Examples of Horizontal Subfigures}
%	\end{figure}
%\end{frame}
%
%\begin{frame}{Example of Horizontal Alignment}
%    % For data collection:
%    
%    Example of Horizontal Alignment of a \texttt{table} and a \texttt{figure}.
%    \begin{center}
%    \begin{minipage}[t]{.65\linewidth}
%    \begin{table}[H]
%    % \renewcommand{\arraystretch}{1.3}
%    \caption{Environment limitations on data collection}
%    \label{tab:env_limit}
%    \centering
%    % \begin{tabular}{m{1.6cm}|c|>{\centering\arraybackslash}m{2cm}|>{\centering\arraybackslash}m{2.3cm}}
%    \begin{tabular}{m{2cm}|c|c|>{\centering\arraybackslash}m{1.5cm}}
%    % \begin{tabular}{c|c|c|c}
%        & Kinect & Stereo & Kinect + Stereo\\
%        \hline
%        Indoor & \cmark & \cmark & \cmark \\
%        \hline
%        Outdoor & \xmark & \cmark & \cmark \\
%        \hline
%        High number of features & \cmark & \cmark & \cmark \\
%        \hline
%        Low number of features & \cmark & \xmark & \cmark 
%    \end{tabular}
%    \end{table}
%    \end{minipage}%
%    \begin{minipage}[t]{.35\linewidth}
%    \vspace{0pt}
%    \centering
%    \includegraphics[width=0.7\textwidth]{images/waist_cam_setup_new.png}
%    \end{minipage}
%    \end{center}
%\end{frame}
%
%\begin{frame}[allowframebreaks]
%\frametitle{Example of resizable equations}
%
%\begin{center}
%\scalebox{1.0}{\parbox{\linewidth}{%
%		\begin{align*}
%		& {\text{min \hskip 6pt}}
%		& & J = \int (a_{real} - \hat{a})^2  \\
%		& \text{subject to}
%		& & \text{human kinematics} \\
%		&&& \text{no collision} \\
%		&&& \text{no falling} 
%		\end{align*}
%}}
%\end{center}
%\end{frame}
%
%\begin{frame}[allowframebreaks]
%\frametitle{Example of Regular Equations}
%    % \begin{equation}
%    %     {}^Ag = {}^AR_B {}^Bg
%    % \end{equation}
%    
%    % \begin{equation}
%    %     V = \frac{{}^Bg \cross {}^Ag}{\norm{{}^Ag}\norm{{}^Bg}}, 
%    %     \theta = \arccos{\frac{{}^Bg \cross {}^Ag}{\norm{{}^Ag}\norm{{}^Bg}}}
%    % \end{equation}
%    
%    \begin{equation}
%        \begin{split}
%        {}^AR_{B}(t_0)=\left[\begin{array}{ccc}
%        1 & 0 & 0 \\
%        0 & 1 & 0 \\
%        0 & 0 & 0
%        \end{array}\right]+
%        \sin (\theta)\left[\begin{array}{ccc}
%        0 & -v_{3} & v_{2} \\
%        v_{3} & 0 & -y_{1} \\
%        -v_{2} & v_{1} & 0
%        \end{array}\right]+ \\
%        (1-\cos (\theta))\left[\begin{array}{ccc}
%        0 & -v_{3} & v_{2} \\
%        v_{3} & 0 & -v_{1} \\
%        -v_{2} & v_{1} & 0
%        \end{array}\right]^{2}
%        \end{split}
%        \end{equation}
%        
%        \begin{align}
%            {}^AR_{B}(t) &= \Delta R {}^AR_{B}(t_0) \\
%            \Delta R &= {}^AR_{B}(t) {}^AR_{B}^T(t_0)
%        \end{align}
%\end{frame}
%
%\begin{frame}[allowframebreaks]
%\frametitle{Example of Video}
%
%	\includemedia[
%	width=\linewidth,
%	totalheight=0.6\linewidth,
%	activate=pageopen,
%	passcontext,  %show VPlayer's right-click menu
%	addresource=images/opensim_video.mp4,
%	flashvars={
%		%important: same path as in `addresource'
%		source=images/opensim_video.mp4
%	}
%	]{\fbox{Click!}}{VPlayer.swf}
%    
%\end{frame}

\begin{frame}[allowframebreaks]
\frametitle{Bibliography}
\printbibliography
\end{frame}

%\begin{frame}[allowframebreaks]
%	\frametitle{References}
%	\bibliographystyle{plainnat}
%	\bibliography{references.bib}
%\end{frame}

\end{document}